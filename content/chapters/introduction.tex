%% Introduction
%%=========================================

\chapter{Introduction}
In this chapter we introduce the scope of this master thesis. The general problem is presented with relevant background and motivation. We also present the goals by defining research questions, define the research methodology, as well as establish how this thesis is structured.

\section{Background and Motivation}
We live in a digital age, and everything around us is getting more and more digital by the day. Data that used to be analogue finds new life in the digital format. Photos, audio, video, and books are just a few types of data that are commonly found digitally these days. The Oxford English Dictionary defines the action or process of digitizing, the task of converting analogue data into digital data, as ``digitization" \cite{misc-oed-digitization}.

In many areas is digitization a process that has been going on for decades. Now that we more often than not assume data is available digitally, countless initiatives have taken it upon themselves to provide digital data that was previously only available in analogue formats. One such project is Bokhylla.no (The bookshelf). Bokhylla.no is a project initiated by the National Library of Norway. It was launched in 2009 and aims to provide online access to literature published in Norwegian. The service will contain about 250 000 books when it is completed in 2017 \cite{misc-nb-digial-library}.

While simply scanning the books will suffice to make the literature available online, other technologies are needed to actually index the content. By indexing, we mean the process of capturing the scanned text and converting it into editable and searchable data. To capture the data, we use a technology called Optical Character Recognition, or OCR for short. OCR is used to convert images of text to machine-encoded text. OCR has many applications, and is in use in many areas today. Book scanning, number plate recognition, handling of checks, passports, as well as assistive technologies for blind and visually impaired users all use some kind of OCR. OCR is an incredible broad area, and an astounding amount of research, approaches and technological solutions are already written and developed in the field. The motivation behind this thesis is to attempt to do character recognition in a bigger problem space using machine learning technologies that are commonly used to solve other problems.

\section{Goals and Research Questions}
Our goal is to use the ``signature" of a character to classify it. By ``signature", we mean that we take characters and only use a horizontal line with a low height to classify it. In Figure \ref{fig:thesis-signature} we have the word ``THESIS". The original characters have a height of 50 pixels, but to do our classification, we only use one pixel of the total height of the characters. The line that defines the signature for this word is highlighted in the figure for illustration purposes.

\begin{figure}[ht]
    \centering
    \includegraphics[width=0.7\textwidth]{fig/chapter1/signature.png}
    \caption{Illustration of a word with a signature with a height of one pixel}
    \label{fig:thesis-signature}
\end{figure}

\red{Give a nice introduction before we introduce the general goal}. The general goal of this thesis is:

\begin{description}
\item[Goal] {\it Find out how well machine learning can predict letters and words in a ambiguous problem space}
\end{description}

\red{Expand on the goal and describe more in details that research questions we want to answer in this thesis.}

\begin{description}
    \item[Research Question 1]{\it derp}
\end{description}

\section{Research Method}
\red{TODO}

\section{Thesis Structure}
\red{TODO}
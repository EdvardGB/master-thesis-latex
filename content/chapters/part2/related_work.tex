%% Related Work
%%=========================================

\chapter{Related Work}
\label{ch:related_work}
As presented in Chapter xx, our problem resides in the area of sequence to sequence learning. Sequence to sequence learning is the task of taking a sequence of input, and map it to our output sequence. While deep neural network can solve many difficult and intricate tasks, it is not capable of solving a sequence to sequence mapping task. However, we can build on existing parts of deep neural network components to build a model that is able to solve such tasks.

%%=========================================

\section{Recurrent Neural Network}
Traditional neural networks assume and inputs and outputs are independent on each other. While this is true for some problems, for other problems there is a direct connection between earlier input and following output. Recurrent Neural Networks were designed around this relationship.

Recurrent Neural Networks, or RNNs, are a class of artificial neural networks with special characteristics. Due to how RNNs have their connections between the units, it is capable of storing an internal state for the network. This state can be used to ``remember" previous outputs and computations. This makes it possible to evaluate input on the basis of previous knowledge and apply this to the output. Recurrent in Recurrent Neural Networks, means that the calculating task is done recurrent for the input.

\red{Figure here}

Because RNNs have a concept of memory and time, it is able to use previous input and feed the knowledge from that back into its internal state. 

\subsection{Use of RNNs}
\red{TODO}

%%=========================================

\section{Variations of RNNs}
\red{TODO}

\subsection{Long Short Term Memory}
Long Short Term Memory, or LSTM for short, is a method that improves on storing information over extended time in recurrent networks. 

Recurrent does, 

\cite{hochreiter1997long}

\subsection{GRU}
\red{TODO}

%%=========================================

\section{Encoder/Decoder}
\red{TODO}

\cite{rocktaschel2015reasoning}
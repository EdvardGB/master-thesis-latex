%% Abstract
%%=========================================

\chapter*{Abstract}
Digital technologies has enabled information to be created, manipulated, disseminated, relocated, and stored with increasing ease. This, in combination with increased storage capacities, as well as cheaper storage units, has made digital formats suitable for preservation and storage. Data that used to be stored in analogue heaps finds new life in digital formats. Google Books, JSTOR, and Bokhylla.no are just a few projects that digitize books, journals, and other written mediums. Converting scanned text into searchable data is done with a technology called optical character recognition, a technology that is used in many applications today. OCR systems have achieving up to 99\% recognition rates when working with clean and well-formatted documents under optimal conditions. However, the results are less promising under sub-optimal conditions, for example when faced with damaged or obfuscated text. \newline

\noindent We propose a new way to recognize machine-written words by using their ``signature'', a small portion of the original text. Our approach to this problem is to consider it as a translation problem, and we attempt to solve it by using state-of-the-art approaches in the area of machine translation. The final, and best performing, model reached an accuracy of over 98\% when recognizing with text written in a single font, and close to 90\% when recognizing with text written in five different fonts under 10\% noise.
%% Abstract
%%=========================================

\chapter*{Abstract}
We find ourselves in the digital age, a

Analogue information finds new life in digital formats. The process of digitization requires several tools to not only copy analogue information, but also make it searchable and editable. Optical Character Recognition is one such technology which is used to capture typed, handwritten, or printed text info machine-encoded text. Despite advances in the field, there is still room for improvement, especially when dealing with damaged of obfuscate text, a relevant concern in real-world usage. In this thesis we purpose a new way to recognize machine-written words by using their ``signature", a small portion of the original text. Our approach to this problem is by considering it as a translation problem, and we attempt to solve it by using state-of-the-art approaches within the area of machine translation. The final, and best performing, model reached an accuracy of over 98\% when dealing with text written in a single font, and close to 90\% when dealing with text written in five different fonts with induced noise.
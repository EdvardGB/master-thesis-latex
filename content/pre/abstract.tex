%% Abstract
%%=========================================

\chapter*{Abstract}
Digital technologies has enabled information to be created, manipulated, disseminated, relocated, and stored with increasing ease. This, in combination with increased storage capacities, as well as cheaper storage units, has made digital formats suitable for preservation and storage. These benefits had lead to countless digitization projects, many of which deals with books and other written mediums. Converting scanned text into searchable data is done with a technology called Optical Character Recognition, a technology that is used in many applications today. OCR systems have reached great results under good conditions when working with clean and well-formatted documents, but the results are less promising under sub-optimal conditions, for example when faced with damaged or obfuscated text. \newline

\noindent We purpose a new way to recognize machine-written words by using their ``signature", a small portion of the original text. Our approach to this problem is to consider it as a translation problem, and we attempt to solve it by using state-of-the-art approaches within the area of machine translation. The final, and best performing, model reached an accuracy of over 98\% when dealing with text written in a single font, and close to 90\% when dealing with text written in five different fonts with induced noise.
%% Sammendrag
%%=========================================

\chapter*{Sammendrag}
\addcontentsline{toc}{chapter}{Sammendrag}
Digitale teknologier har gjort det eklere for informasjon å bli laget, manipulert, formidlet, flyttet og larget. Dette, i kombinasjon med økt lagringskapasitet, og billigere lagringsenheter, har gjort digitale formater passende for oppbevaring og lagring. Informasjon som pleide å bli larget i analoge hauger finner nytt liv i digitale formater. Google Books, JSTOR og Bokhylla.no er bare noen få prosjekter som digitaliserer bøker, journaler og andre skrevne medier. Konvertering av skannet test til søkbar data er gjort med en teknologi som kalles optisk tegngjenkjenning, en teknologi som brukes i mange områder i dag. Optisk tegngjenkjenningssystem har oppnådd opp til 99\% gjennkjenningsrate på rene og velformatterte dokumenter under optimale forhold. Resultatene er mindre lovende under suboptimale forhold, for eksamel når teksten er skadet eller tilslørt. \newline

\noindent Vi foreslår en ny måte å gjøre gjennkjenning av ord på, ved å bruke dets ``signatur'' en liten del av den originale teksten. Vi betrakter problemet som et oversettingsproblem, og vi forsøker å løse det ved å bruke ``state-of-the-art'' metoder innen maskinoversetting. Den siste, og beste, modellen vår gjenkjente med en treffsikkerhet på over 98\% på tekst skrevet i en skrifttype, og nærmere 90\% på tekst skrevet i fem forskjellige skrifttyper med 10\% støy.
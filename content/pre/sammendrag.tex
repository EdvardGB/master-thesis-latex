%% Sammendrag
%%=========================================

\chapter*{Sammendrag}
\addcontentsline{toc}{chapter}{Sammendrag}
Optisk tegngjenkjenning er en teknologi som brukes for å konvertere skannet tekst til søkbar data. Optisk tegngjenkjenningssystemer har oppnådd nærmere 99\% gjennkjenningsrate på rene og velformaterte dokumenter under optimale forhold. Resultatene er derimot mindre lovende under suboptimale forhold, for eksempel når teksten er skadet eller tilslørt. \newline

\noindent Vi foreslår en ny metode for gjenkjenning av ord som er tilslørt på en bestemt måte. Denne gjenkjenningen gjøres ved å bruke ``signaturen'' til ordet, som er en liten del av den originale teksten. Vi betrakter problemet som et oversettingsproblem, og vi forsøker å løse det ved å bruke ``state-of-the-art'' metoder innen maskinoversettelse. Tre modeller ble utvikler som et resultat av forskningen utført i denne masteroppgaven. To av disse var basert på encoder-decoder-rammeverket for sekvens-til-sekvens-prediksjon. Den beste modellen gjenkjente med en treffsikkerhet på over 98\% på tekst skrevet i en skrifttype, og nærmere 90\% på tekst skrevet i fem forskjellige skrifttyper under 10\% støy.
%%=========================================

\section{Personal Experience and Motivation}
The author of this master thesis studies informatics with artificial intelligence as the chosen field of study. Naturally, leading up to the writing of this thesis, he had several courses that were directly related to artificial intelligence. These were both introduction courses, as well as intermediate, and more advanced ones. The course content included implementing a wide range of classic AI algorithms, principles of logic and cognitive sciences, representation of uncertain knowledge, various methods related to decision making and learning/adaptive systems, as well as methods in general machine learning and some underlying basic mechanisms for various specific methods. 

Artificial neural networks was one computational approach that several courses contained. The author was taught the principles of artificial neural networks by implementing perceptrons, and constructing simple artificial neural networks with feed forward and backpropagation. Later, more advanced networks were used to classify characters with the MNIST dataset, and to play and ``learn" the game 2048. Even more advanced networks in the class of recurrent neural networks were used in a variant of the Wumpus world problem, and in a game problem where we attempted to ``capture" or ``avoid" a set of moving objects given a set of constraints.

The problem presented in this thesis resolves around the use of artificial neural networks to do an offspring of image classification, using sequence prediction as a type of language translation. The techniques necessary to dive into a problem like this had, to some degree, been introduced in some of the courses the author previously had. The combination of personal experience in the area of machine learning, and motivation to use it to experiment with new approaches, was what laid the foundation for this thesis.

Some of the motivation was also to look at new applicabilities of artificial neural networks. Artificial neural networks has since the start of the 21st century gained a lot of attraction. With the increase computational powers of affordable computers, as well as new tools and libraries that are both user friendly and highly applicable, machine learning is more accessible now than ever. 

\red{More here}

%%=========================================

\section{Process}
The process of iteratively developing our model start with minimal background information and knowledge. We go through several phrases of researching relevant background theory, testing certain approaches, evaluation our results and improve on our solution. That process from the beginning to end is 

\subsection{Preliminary Research}
Before work began on our model, we did extensive preliminary research on problems that were similar in nature. This was done to gain knowledge about published literature, existing models and establishing what type of work had already been done in relevant areas. This phase created the foundation for which we could to the next steps of our process. As OCR and image recognition are very wide areas, a lot of research is both done and published. It was impossible to read everything, but we tried to focus on highly acclaimed papers, and papers that had a general approach to the problems we were trying to solve. A more extensive look at the preliminary research can be found in Chapter x, where we look at related work that was found in both this step, and proceeding steps.

\subsection{Iterative Development Process}
Following the preliminary research, we followed roughly the cycles as described in \ref{methodology-design-and-creation-approach}. First, on the basis of the research we had conducted up to this point, we came up with a tentative plan for how we could solve our problem. We then implemented this model as a prototype, and tested it on our problem. The results were used to determine how we would proceed our research, and where we would look to find inspiration for our next tentative plan.

During this process we tested a few approaches that could not handle our problem due to a series of reasons, as explained in Chapter xx, but they all helped us steer the course towards our final model and its architecture.

\subsection{Evaluating Our Final Model}
After a few round of the iterative development process, we ended up with a model that we felt was closing in on a solution. While no model is perfect, and ours surly has it flaws, we were satisfied with how it worked given our limitations and the nature of the problem. Because this model was a part of an iterative development process, every part of the architecture had been tested in various ways, and we were confident that we could argue about every choice we had made. Evaluation was done by using this model and running it on several problems, recording the results and evaluating it compared to various metrics.

\subsection{Looking Into Future Work}
As a part of the evaluation, we look at the results our model produced, and tried to find out why we got the results we got. Identifying our weaknesses and finding potential for future work lay the foundation for continued research. This is done in Chapter xx.
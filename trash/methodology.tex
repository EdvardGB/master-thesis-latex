%%=========================================

\section{Personal Experience and Motivation}
The author of this master thesis studies informatics with artificial intelligence as the chosen field of study. Naturally, leading up to the writing of this thesis, he had several courses that were directly related to artificial intelligence. These were both introduction courses, as well as intermediate, and more advanced ones. The course content included implementing a wide range of classic AI algorithms, principles of logic and cognitive sciences, representation of uncertain knowledge, various methods related to decision making and learning/adaptive systems, as well as methods in general machine learning and some underlying basic mechanisms for various specific methods. 

Artificial neural networks was one computational approach that several courses contained. The author was taught the principles of artificial neural networks by implementing perceptrons, and constructing simple artificial neural networks with feed forward and backpropagation. Later, more advanced networks were used to classify characters with the MNIST dataset, and to play and ``learn" the game 2048. Even more advanced networks in the class of recurrent neural networks were used in a variant of the Wumpus world problem, and in a game problem where we attempted to ``capture" or ``avoid" a set of moving objects given a set of constraints.

The problem presented in this thesis resolves around the use of artificial neural networks to do an offspring of image classification, using sequence prediction as a type of language translation. The techniques necessary to dive into a problem like this had, to some degree, been introduced in some of the courses the author previously had. The combination of personal experience in the area of machine learning, and motivation to use it to experiment with new approaches, was what laid the foundation for this thesis.

Some of the motivation was also to look at new applicabilities of artificial neural networks. Artificial neural networks has since the start of the 21st century gained a lot of attraction. With the increase computational powers of affordable computers, as well as new tools and libraries that are both user friendly and highly applicable, machine learning is more accessible now than ever. 

\red{More here}
%% Packages
%%=========================================

%% Translates various standard and other input encodings into a LaTeX internal language
\usepackage[utf8]{inputenc}

%% Daim said so
\usepackage{lmodern}

%% Correct English hyphenation
\usepackage[english]{babel}

%% ??
\usepackage{fourier}

%% Make it possible to remove Figure xx: from caption text
\usepackage{caption}

%% ??
\usepackage{marvosym}

%% Inclusion of graphics
\usepackage{graphicx}

%% Compiling lists of glossaries
\usepackage[xindy,nonumberlist]{glossaries}

%% Better geometry
\usepackage{geometry}

%% Floating of figures
\usepackage{float}

%% ??
\usepackage{rotating}

%% Good math mode
\usepackage{amsmath}

%% ??
\usepackage{booktabs}

%% ??
\usepackage{url}

%% Biblopgraphy
\usepackage{natbib}

%% Sub captions
\usepackage{subcaption}

%% ??
\usepackage{setspace}

%% Handles cross-referencing commands to produce hyper text links in the document
\usepackage[colorlinks=true, pdfstartview=FitV, linkcolor=black, citecolor=black, urlcolor=black]{hyperref}


%% Pseudocode
\usepackage[noend]{algpseudocode}
\usepackage{algorithm}

%% Used to create "correct" empty pages
\usepackage{afterpage}

%% Used for note box
\usepackage{blindtext}
\usepackage[most]{tcolorbox}

%% Plots
\usepackage{pgfplots}
\usetikzlibrary{backgrounds}

%% Degree
\usepackage{gensymb}

%% Multirow
\usepackage{multirow}

%% Plot stuff
\usepackage{filecontents}